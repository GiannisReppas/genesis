\documentclass[a4paper,10pt]{article}
\usepackage[utf8]{inputenc}
\usepackage{hyperref}

\setlength{\parindent}{0pt}
\setlength{\parskip}{1em}

\title{Documentation for Developing Genesis}
\author{Lucas Czech}

\begin{document}
\maketitle

\section{Introduction}
\label{sec:Introduction}

This document contains information for developers who want to contribute to Genesis: Coding conventions, useful sources and resources, explanations of design decisions, etc. (well, not yet all of this, but it hopefully will soon)

\section{Getting Started}
\label{sec:GettingStarted}

\subsection{Architecture}
\label{sec:GettingStarted:sub:Architecture}

Genesis in mainly a framwork/library, thus most code is structured in classes for different purposes, which are loosely coupled.

The library code itself is found in \verb|./lib| and additional source (code for Python bindings etc) in \verb|./src|

\subsection{Organization}
\label{sec:GettingStarted:sub:Organization}

Classes belonging to a certain bigger topic are organized in feature modules, e.g. everything for working with trees or alignments.

The classes dealing with data structures (Tree, Alignment, ...) are meant to be stand-alone, and general purpose. They do the stuff that is inherent to them, but not more. A tree does not know about likelihood or evolutionary placement. Those are matters for different classes and modules.

Data input and output is also a matter for single purpose classes. They are mainly static, take a filename or input string and a data structure class (e.g. Tree) and read/write the data. They are called processors, for example \verb|FastaProcessor|.

Logging can be done using the macros defined in \verb|./utils/logging|: Use \verb|LOG_INFO << "message";| and others whenever you want user output. Do not use cout or printf directly!

There are different levels of logging:
\begin{itemize}
    \item \verb|LOG_ERR|: Something went wrong, program needs to be exited. Should be rare.
    \item \verb|LOG_WARN|: Something went wrong, we need to abort the step (read a file, parsing error in a file).
    \item \verb|LOG_INFO|: Stuff that helps understanding, what the program does (dumps of data, read or write a file, ...)
    \item \verb|LOG_DBG|: Useful in development for outputting debug messages.
    \item \verb|LOG_DBGx|: Replace \verb|x| by a number between 1 and 4. This is a debug message with indention -- useful within loops!
\end{itemize}

Many classes use a function \verb|Dump()| that returns a human-readable string containing the information and data in the class. This is useful for debug.

Also, classes with complex data structures can introduce a \verb|Validate()| function that checks the consistency of the data (are all pointers valid, are the index numbers valid, etc) and returns a bool true if so.

Use assertions regularly to document program invariants. And write an explicit comment explaining why this is an invariant.

% naming
% 
% "from" and "to" for string based data and file access (also from string instead of parse string! parse is just the name of the class for the actual parser, not for the function called to populate an object or to process strings/tokens)
% load save for binary file access

% "process" for doing some stuff on input like strings or tokens. not tokenize, parse or so. keep it simple (and consistent)!

% important
% 34. Prefer composition to inheritance.
% 37. Public inheritance is substitutability. Inherit, not to reuse, but to be reused.
% 
% 44. Prefer writing nonmember nonfriend functions.

% use assert frequently!
% however, assert is only used to control if program invariants are actually invariant. this means, only check something that MUST ALWAYS be true -- this way it only fails if there is a bug in your code, which is then easier to track down. NOT for "very rare situations that should not actually appear"!
% example: put an assert(false) in a place that can only be reached when you forgot to handle a special case. this way, you will immediately know that you forgot this case once it appears.
% also, make an elaborate comment for each assert explaining why it actually has to be an invariant. this helps other people understanding your code.

% error messages: keep them short and to the scope -- leave the explanation of the scope to a higher level function.
% for example, when lexing some string fails, simply give the error "missing parenthesis at ..." and leave the part "lexing failed. message: ..." to the function that called the lexer.

\section{GIT Repository}
\label{sec:GitRepository}

For each major feature, create a branch named "feat/name" using \verb|git branch -t feat/name|
To see a list of existing branches, use \verb|git branch --list|

The master branch is mainly for merging in from the feature branches.

This yields the following typical workflow:
\begin{verbatim}
    git checkout feat/tree
    git merge master
    ... do work ...
    git commit -am "Achieved something today!"
    git checkout master
    git merge feat/tree
\end{verbatim} 

Commit regularly and often -- and use single commits for single purposes.

Commit only when you achived something -- the code compiles, you finished a function, etc.

Give each commit a fitting short description. Use an empty line followed by more details if necessary.

% bug fixes and similar short-term changes are done on brachnches named
% "fix/description"
% 
% commit messages: use "new: " and "chg: " in front of the first line to mark
% those messages that are interesting for the changelog.
% more precicesly: add marks new features that were not there before and are now
% available in the api, chg marks changes that affect the api without adding
% functionality.

\section{Coding Conventions}
\label{sec:CodingConventions}

Coding conventions for genesis mainly follow the Google C++ Style Guide: \href{http://google-styleguide.googlecode.com/svn/trunk/cppguide.html}{http://google-styleguide.googlecode.com/svn/trunk/cppguide.html}
\\
Read carefully!

The most important rules are the chapters on \textbf{headers}, \textbf{classes}, \textbf{naming}, \textbf{comments} and \textbf{formatting}.

Key differences:
\begin{itemize}
    \item Use a line length of max 100 chars.
    \item Ident with four spaces per level.
\end{itemize}

The block start brace \verb|{| appears in a single line for classes and functions, but at the end of the line in normal code (loops, conditions, etc).

\subsection{Classes, Files and Naming}
\label{sec:CodingConventions:sub:ClassesFilesNaming}

As genesis is mainly a framwork/library, everything goes into a class. There is almost no need for standalone functions.

To create a new class and its basic structure, you can use \verb|./tools/create_class.sh|.

Usually, the class name startes with the module name: \verb|TreeEdge|, \verb|TreeNode| etc. Classes for processing files start with the name of the format: \verb|FastaProcessor|.

Files are named in lower case, single words separated by underscores.

The file structure follows the feature/module approach: Everything belonging to a special module goes in the same directory, e.g. all files related to trees are in a folder/module named tree. 

If the file name becomes long, the first part (module name) might be dropped (however not the class name itself, just the file name): \verb|./lib/placement/simulator.hpp| contains class \verb|PlacementSimulator|.

Usually, one file contains one class. However, small helper classes for another class can also be put in the same file, for example the lexer for a file format.

\subsection{Comments}
\label{sec:CodingConventions:sub:Comments}

Use comments frequently -- to state intention, not to re-write your code.

Your code should be understandable by just reading the comments.

Use in-line comments \verb|//| at the beginning of a functional block (some statements with a specific purpose) to explain this purpose. End the block with a blank line.
Example:
\begin{verbatim}
    // count number of leaves
    size_t cnt = 0;
    for (...) {
        if (is_leaf) {
            ++cnt;
        }
    }
    
    // use number of leaves to do something
    ...
\end{verbatim}

Use doc-block comments (starting with \verb|/**| double asterisks, every following line starting with a single asterisk \verb|*|) at the beginning of classes and functions. This will be used by doxygen to automatically generate the API documentation.

Use special comment blocks to section your code for the reader:
\begin{verbatim}
// ===========================================================
//     Tree
// ===========================================================

/**
 * @brief Comment on Tree class.
 */
class Tree
{
public:

    // -----------------------------------------
    //     Typedefs
    // -----------------------------------------
    
    typdef some thing;
    
    ...
}
\end{verbatim}
See existing code for more examples on this.

Use comments starting with \verb|TODO| or \verb|FIXME| for unfinished business. Those will be extracted into a \verb|TODO| file by the script \verb|./tools/extract_todos.sh|

\end{document}
